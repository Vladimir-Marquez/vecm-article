% Options for packages loaded elsewhere
\PassOptionsToPackage{unicode}{hyperref}
\PassOptionsToPackage{hyphens}{url}
%
\documentclass[
]{article}
\usepackage{amsmath,amssymb}
\usepackage{iftex}
\ifPDFTeX
  \usepackage[T1]{fontenc}
  \usepackage[utf8]{inputenc}
  \usepackage{textcomp} % provide euro and other symbols
\else % if luatex or xetex
  \usepackage{unicode-math} % this also loads fontspec
  \defaultfontfeatures{Scale=MatchLowercase}
  \defaultfontfeatures[\rmfamily]{Ligatures=TeX,Scale=1}
\fi
\usepackage{lmodern}
\ifPDFTeX\else
  % xetex/luatex font selection
\fi
% Use upquote if available, for straight quotes in verbatim environments
\IfFileExists{upquote.sty}{\usepackage{upquote}}{}
\IfFileExists{microtype.sty}{% use microtype if available
  \usepackage[]{microtype}
  \UseMicrotypeSet[protrusion]{basicmath} % disable protrusion for tt fonts
}{}
\makeatletter
\@ifundefined{KOMAClassName}{% if non-KOMA class
  \IfFileExists{parskip.sty}{%
    \usepackage{parskip}
  }{% else
    \setlength{\parindent}{0pt}
    \setlength{\parskip}{6pt plus 2pt minus 1pt}}
}{% if KOMA class
  \KOMAoptions{parskip=half}}
\makeatother
\usepackage{xcolor}
\usepackage[margin=1in]{geometry}
\usepackage{graphicx}
\makeatletter
\newsavebox\pandoc@box
\newcommand*\pandocbounded[1]{% scales image to fit in text height/width
  \sbox\pandoc@box{#1}%
  \Gscale@div\@tempa{\textheight}{\dimexpr\ht\pandoc@box+\dp\pandoc@box\relax}%
  \Gscale@div\@tempb{\linewidth}{\wd\pandoc@box}%
  \ifdim\@tempb\p@<\@tempa\p@\let\@tempa\@tempb\fi% select the smaller of both
  \ifdim\@tempa\p@<\p@\scalebox{\@tempa}{\usebox\pandoc@box}%
  \else\usebox{\pandoc@box}%
  \fi%
}
% Set default figure placement to htbp
\def\fps@figure{htbp}
\makeatother
\setlength{\emergencystretch}{3em} % prevent overfull lines
\providecommand{\tightlist}{%
  \setlength{\itemsep}{0pt}\setlength{\parskip}{0pt}}
\setcounter{secnumdepth}{-\maxdimen} % remove section numbering
\usepackage{bookmark}
\IfFileExists{xurl.sty}{\usepackage{xurl}}{} % add URL line breaks if available
\urlstyle{same}
\hypersetup{
  pdftitle={articulo},
  pdfauthor={Vladimir Marquez Stone},
  hidelinks,
  pdfcreator={LaTeX via pandoc}}

\title{articulo}
\author{Vladimir Marquez Stone}
\date{2025-08-27}

\begin{document}
\maketitle

library(readxl) library(urca) library(vars) library(tseries)
library(ggplot2) library(dplyr) library(tidyr)

\section{--- 1) Reading data}\label{reading-data}

\section{Expecting columns: year, profit, surplusvalue, occ (rename to
what you
have)}\label{expecting-columns-year-profit-surplusvalue-occ-rename-to-what-you-have}

data \textless- read\_xlsx(``mydata.xlsx'', sheet = ``Sheet1'')
summary(data)

\section{--- 2) Building time series}\label{building-time-series}

\section{Example for annual data from 1940 to
2023:}\label{example-for-annual-data-from-1940-to-2023}

Y \textless- ts(data{[}, c(``profit'',``surplusvalue'',``occ''){]},
start = min(data\$year), frequency = 1)

\section{--- 3) Selecting lags}\label{selecting-lags}

sel \textless- VARselect(Y, lag.max = 6, type = ``const'')
sel\(selection
K <- as.numeric(sel\)selection{[}``AIC(n)''{]}) \# pick AIC/HQ/SC per
your preference summary(sel)

\section{--- 4) Performing Johansen
test}\label{performing-johansen-test}

joh \textless- ca.jo(Y, type = ``trace'', ecdet = ``const'', K = 2, spec
= ``transitory'') summary(joh) \# Decide cointegration rank r from the
trace tests: r \textless- 1

\section{--- 5) Estimating VECM}\label{estimating-vecm}

vecm\_fit \textless- cajorls(joh, r = r) alpha \textless-
\href{mailto:joh@V}{\nolinkurl{joh@V}}{[}, 1:r, drop = FALSE{]} beta
\textless- \href{mailto:joh@W}{\nolinkurl{joh@W}}{[}, 1:r, drop =
FALSE{]} alpha; beta

\section{Extracting coefficients}\label{extracting-coefficients}

coef\_table \textless- coef(vecm\_fit\$rlm) print(coef\_table)

\section{--- 6) Converting to VAR form and compute
IRFs}\label{converting-to-var-form-and-compute-irfs}

vec\_as\_var \textless- vec2var(joh, r = r)

set.seed(123) ir \textless- irf(vec\_as\_var, n.ahead = 20, boot = TRUE,
runs = 1000, ci = 0.95, ortho = TRUE)

\section{--- 7) IRF plots}\label{irf-plots}

make\_irf\_df \textless- function(ir\_obj) \{ out \textless- list()
impulses \textless- names(ir\_obj\(irf)
  for (imp in impulses) {
    mat   <- ir_obj\)irf{[}{[}imp{]}{]} lower \textless-
ir\_obj\(Lower[[imp]]
    upper <- ir_obj\)Upper{[}{[}imp{]}{]} for (resp in colnames(mat)) \{
out{[}{[}length(out) + 1{]}{]} \textless- data.frame( h = 0:(nrow(mat) -
1), irf = mat{[}, resp{]}, lower = lower{[}, resp{]}, upper = upper{[},
resp{]}, impulse = imp, response = resp ) \} \} bind\_rows(out) \}

ir\_df \textless- make\_irf\_df(ir)

ggplot(ir\_df, aes(x = h, y = irf)) + geom\_ribbon(aes(ymin = lower,
ymax = upper), alpha = 0.2) + geom\_line() + facet\_grid(response
\textasciitilde{} impulse, scales = ``free\_y'') + labs(x = ``Horizon'',
y = ``IRF'', title = ``IRFs from VECM (converted to VAR form)'',
subtitle = paste0(``K (levels lags) ='', K, ``, rank r ='', r, ``,
ordering ='', paste(colnames(Y), collapse = '' → ``))) +
theme\_minimal(base\_size = 12)

summary(joh) summary(irf)

\section{--- 8) CUMULATIVE IRFs}\label{cumulative-irfs}

irc \textless- irf(vec\_as\_var, n.ahead = 20, boot = TRUE, runs = 500,
ci = 0.95, ortho = TRUE, cumulative = TRUE) plot(irc)

\section{--- 9) Forecast Error Variance
Decomposition}\label{forecast-error-variance-decomposition}

fe \textless- fevd(vec\_as\_var, n.ahead = 20) plot(fe)

\end{document}
